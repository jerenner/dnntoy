\pdfoutput=1 % only if pdf/png/jpg images are used
\documentclass{JINST}

\usepackage{amsmath,amssymb}
\usepackage[numbers,sort]{natbib}
%\hypersetup{colorlinks, citecolor=blue, linkcolor=blue, filecolor=blue, urlcolor=red}

\title{Title: NEXT and DNNs}

\author{Author 1$^a$\thanks{Corresponding author.},
Author 2$^b$\\
\llap{$^a$}Instituto de F\'isica Corpuscular (IFIC), CSIC \& Universitat de Val\`encia,\\ 
Calle Catedr\'atico Jos\'e Beltr\'an, 2, 46980 Paterna, Valencia, Spain\\
\llap{$^b$}University of Texas at Arlington\\
 701 S. Nedderman Drive, Arlington, TX 76019, USA\\
E-mail: \email{correspodingauthor@ific.uv.es}}

\bibliographystyle{unsrtnat}
\input{commands}

\abstract{Abstract}

\keywords{Neutrinoless double beta decay; deep neural networks; TPC; high-pressure xenon chambers;  Xenon; NEXT-100 experiment}

\begin{document}

\section{Introduction}\label{sec:intro}
\noindent Here introduce NEXT, track reconstruction in xenon TPCs, and discuss the difference between a signal and background event.

\section{Simulation methods}
Describe the different simulations run.

\subsection{Simple Monte Carlo}
Describe the toy MC (dE/dx, Gaussian MS) and voxelization procedure.

\subsection{GEANT Box Monte Carlo}
Describe the MAGBOX simulation and Paolina voxelization.

\subsection{NEXT-100 Monte Carlo}
Describe the NEXT-100 Monte Carlo.  Note that out of the three Monte Carlo simulations discussed, this simulation most closely matches the actual experiment, though it still operates
directly on Monte Carlo hits and is still not a full simulation with realistic track reconstruction.

\section{Topological Analysis}
Describe the standard topological analysis procedure applied to the NEXT-100 data.

\section{Event classification with a deep neural network (DNN)}
\subsection{Classification procedure}
Describe the data preparation and DNN analysis procedure (input images to Caffe via DIGITS).  Note selection of the GoogLeNet and the use of three 2D projections.

\subsection{Analysis of NEXT-100 Monte Carlo}
Discuss the results obtained with GoogLeNet for NEXT-100 Monte Carlo, to be compared with the topological analysis.  Show the cases of 2x2x2 and 10x10x10 voxels.  This comparison is
the main result of the paper.

\subsection{Evaluating the DNN analysis}
Here we show the DNN analysis for the different Monte Carlo runs, explaining how we arrived at the conclusion that delta rays and energy fluctuations are mainly responsible for the
loss of accuracy.

\section{Conclusions}
DNN analysis with just three projections seems to be capable of outperforming conventional analysis.  Discuss future plans to develop an optimized DNN analysis and speculate
on possible improvements.

%\begin{figure}[!htb]
%\centering
%\includegraphics[width= 0.95\textwidth]{fig/SoftAsymmetric_bound.pdf}
%\caption{Principle of operation of an asymmetric HPXe TPC with EL readout (figure from \cite{MartinAlbo_thesis}).} \label{fig.SS}
%\end{figure}

%\appendix
%\section{A Lowpass FIR Filter}\label{app:FIR}

\acknowledgments

This work was supported by the European Research Council under the Advanced Grant 339787-NEXT and the Ministerio de Econom\'{i}a y Competitividad of Spain under Grants CONSOLIDER-Ingenio 2010 CSD2008-0037 (CUP), FPA2009-13697-C04-04, FPA2009-13697-C04-01, FIS2012-37947-C04-01, FIS2012-37947-C04-02, FIS2012-37947-C04-03, and FIS2012-37947-C04-04.  JR acknowledges support from a Fulbright Junior Research Award.

\bibliography{dnnext}


\end{document}